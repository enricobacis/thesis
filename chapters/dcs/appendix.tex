\chapter[Decentralized Cloud Storage --- Proofs of Theorems]{Securing Resources in Decentralized Cloud Storage\\--- Proofs of Theorems}
\label{dcs:sec:proofs}

\begin{proof}
	\hfill\\
	Let us assume, by contradiction, the existence of a
    $(\nk,\nr)$-allocation for a resource split into $\ns = \nk <
    \nk +1$ slices. Given \ns\ different slices, no more than
    \ns\ nodes can be used to store one replica of the
    slices. Since \ns=\nk, the allocation function is storing the
    whole resource using at most \nk\ nodes. Therefore, it is not
    \nk-protected.  Note that $\nk + 1$ slices are sufficient to
    define a $(\nk,\nr)$-allocation. The \nr-replication
    requirement is easily satisfied by replicating \nr\ times each
    of the $\nk + 1$ slices. The \nk-protection requirement is
    satisfied by storing each slice to a different node. Hence,
    for each replica of the resource, each coalition of \nk\ nodes
    misses one slice (the one stored at the $(\nk+1)$-th
    node).
    {\hfill $\square$}
\end{proof}

\medskip

\begin{proof}
	\hfill\\
	Since there are $\nr \cdot (\nk+1)$ slices to be stored, a
    $(\nk,\nr)$-allocation cannot use more than $\nr \cdot
    (\nk+1)$ nodes.  However, a $(\nk,\nr)$-allocation with
    $\ns=\nk+1$ cannot use less than $\nr \cdot (\nk+1)$
    nodes. Let us consider the case where slices are not
    replicated (i.e., $\nr=1$), since the same discussion applies
    to each replica of the resource. Assume, by contradiction,
    that a $(\nk,\nr)$-allocation stores more than one slice at
    one of the nodes. If the function adopts $\nn=\nk+1$ nodes,
    there will be at least one node that does not store any slice
    (which is equivalent to say that $\nn \leq \nk$) as the number
    of slices is $\nk+1$.  Then, \nk\ nodes store all the slices
    composing the resource, thus violating \nk-protection.
    {\hfill $\square$}
\end{proof}

\medskip

\begin{proof}
	\hfill\\
	To guarantee \nr-replication, each slice should be stored at
    (at least) \nr\ nodes. If allocation function \f\ is
    \nk-protected, for each coalition \setnode{i} of \nk\ nodes,
    there exists at least a slice \shard{j} that is not stored at
    any of the nodes in \setnode{i}. To guarantee that \shard{j}
    has \nr\ copies, there must exist at least \nr\ additional
    nodes that store \shard{j}, and \nn\ should be at least equal
    to $\nk + \nr$.  Note that $\nk + \nr$ nodes are sufficient to
    define a $(\nk,\nr)$-allocation.  Consider, as an example,
    $\ns=\binom{\nk+\nr}{\nk}$ slices, the set
    $\setnode{1},\ldots,\setnode{\ns}$ of possible coalitions of
    \nk\ nodes, and allocation function \f\ that assigns the
    $i$-th slice to all the nodes in \Nodes, but the ones in the
    $i$-th coalition: $\f(\shard{i})=\Nodes\ \setminus
    \{\setnode{i}\}$. Function \f\ is \nk-protected, since each
    coalition \setnode{i} cannot access slice \shard{i}, and
    \nr-replicated, since slice \shard{i} is stored at each node
    $\node{i} \in \Nodes\ \setminus \{\setnode{i}\}$, then at $\nn
    - \nk = \nr$ nodes.
    {\hfill $\square$}
\end{proof}

\medskip

\begin{proof}
	\hfill\\
	An allocation function \f\ is \nk-protected if the set of
    slices stored at any coalition \setnode{i} of \nk\ nodes is
    not complete (i.e., at least one slice is missing). To
    guarantee that \f\ is an allocation function each slice should
    be stored at least on one node. If each coalition misses a
    different slice, we are minimizing the number of slices
    necessary to define an allocation function \f. Indeed, since
    $\nn > \nk$, the missing slice for each coalition \setnode{i}
    can be stored at the nodes in $\Nodes \setminus \setnode{i}$,
    as they will miss another slice. Since there are
    $\binom{\nk+\nr}{\nk}$ possible coalitions of \nk\ nodes in
    \Nodes, \ns\ should be at least $\binom{\nk+\nr}{\nk}$ to
    guarantee that each coalition misses a different slice.
    Assume, by contradiction, that $\nr=1$,
    $\ns=\binom{\nk+\nr}{\nk}-1$, and that \f\ is
    \nk-protected. In this case, two coalitions \setnode{i} and
    \setnode{j} will miss the same slice \shard{x}. That is, there
    are at least $\nk+1$ nodes (the ones in $\setnode{i} \cup
    \setnode{j}$) missing \shard{x}. However, if $\nn = \nk +1$
    then \f\ cannot be an allocation function, since no node in
    \Nodes\ stores \shard{x}. (The same reasoning applies with
    larger values for \nr.)
    {\hfill $\square$}
 \end{proof}

\medskip

\begin{proof}
	\hfill\\
	{\em 1)\/} Since each coalition of \nk\ nodes should not be able to
	reconstruct the resource, it should miss at least one slice. The
	number of slices used by the allocation function is
	$\ns=\binom{\nk+\nr}{\nk}$, which is sufficient for each coalition to
	miss at least one slice. In fact, the number of possible coalitions of
	\nk\ nodes is $\binom{\nk+\nr}{\nk}$. Let us assume, by contradiction,
	that two coalitions \setnode{i} and \setnode{j} miss the same slice
	\shard{x}. Therefore, there are $\nk+1$ nodes
	(\setnode{i}$\cup$\setnode{j}) that do not store \shard{x}. However,
	in this case \shard{x} would be stored at $\nn - (\nk +1)=\nr-1$
	nodes. Hence, the allocation function would not be
	\nr-replicated.
	
	\medskip
	
	\noindent {\em 2)\/} By definition of
	$(\nk,\nr)$-allocation with $\nn = \nk+\nr$ nodes and
	$\ns=\binom{\nn}{\nk}$ slices, each coalition of \nk\ nodes misses a
	different slice. Let  us consider two coalitions \setnode{i} and
	\setnode{j}  that differ in one node only. Coalition \setnode{j}
	misses one slice, \shard{j}, while \setnode{i} misses \shard{i}. Since
	\setnode{j} misses one slice only, it stores \shard{i}. Hence,
	\setnode{i}$\cup$\setnode{j} includes $\nk+1$ nodes and stores all the
	slices composing the resource.
	{\hfill $\square$}
\end{proof}

\medskip

\begin{proof}
	\hfill\\
	{\em \PF)\/} The probability to obtain back the original plaintext resource
	corresponds to the probability that $\nk+1$ nodes, each storing a
	different slice, do not fail. Since \pf\ is the probability that a
	node fails, the probability that at least one of the \nr\ replicas of
	a slice is available is $(1-(\pf)^{\nr})$, with $(\pf)^{\nr}$ the
	probability that all \nr\ replicas of the slice are unavailable. 
	The probability to obtain back all the $\nk+1$ slices, 
	and then also the original plaintext resource, is then
	$(1-(\pf)^{\nr})^{\nk+1}$ and $(1-(1-(\pf)^{\nr})^{\nk+1})$ is the
	probability that the resource cannot be decrypted.
	
	\medskip
	
	\noindent {\em  \PC)\/} The probability that the resource is exposed 
	is the probability that all the slices are
	compromised, meaning that $\nk+1$ nodes are malicious.
	Since $1-\pc$ is the probability that a node is not compromised and
	each slice has \nr\ replicas, the probability that at least one
	replica is exposed is $(1-(1-\pc)^{\nr})$, with
	$(1-\pc)^{\nr}$ the probability that all \nr\ replicas of a slice
	are not compromised. We can then conclude that the probability that
	all $\K+1$ slices are exposed is $(1 - (1-\pc)^{\R})^{\K+1}$.
	{\hfill $\square$}
\end{proof}

\medskip

\begin{proof}
	\hfill\\
	{\em \PF)\/} To obtain back the original plaintext
	resource, we need the slices stored on any combination of $\nk+1$
	nodes, that is, $\nk+1$ nodes must not fail. A resource therefore
	becomes unavailable when any combination of \nr\ or more nodes
	fail. The probability that $i$ nodes fail, with
	$i=\nr,\ldots,\nk+\nr$, and $\nk+\nr-i$ nodes do not fail is
	equal to $ (\pf)^i (1 - \pf)^{{\K}+{\R}-i}$.  Since the number of
	combinations of $i$ nodes out of $\nk+\nr$ is
	${{\K+\R}\choose{i}}$, the probability that a resource is
	unavailable is $\PF= \sum\limits_{i={\R}}^{{\K}+{\R}}
	{{\K+\R}\choose{i}} (\pf)^i (1 - \pf)^{{\K}+{\R}-i}$.
	
	\medskip
	
	\noindent {\em  \PC)\/} A coalition can compromise the confidentiality
	of the resource whenever it involves any combination of $\nk+1$
	nodes, that is, at least $\nk+1$ nodes must be compromised. The
	probability that $i$ nodes are compromised, with
	$i=\nk+1,\ldots,\nk+\nr$, and $\nk+\nr-i$ nodes are not compromised
	is equal to $(\pc)^i (1 - \pc)^{{\K}+{\R}-i}$.  Since the number of
	combinations of $i$ nodes out of $\nk+\nr$ is the binomial
	coefficient ${{\K+\R}\choose{i}}$, we can conclude that the
	probability that any combination of at least $\K+1$ nodes are
	compromised in a collection of $\R+\K$ nodes is $\PC=
	\sum\limits_{i={\K}+1}^{{\K}+{\R}} {{\K+\R}\choose{i}} (\pc)^i (1 -
	\pc)^{{\K}+{\R}-i}$.
	{\hfill $\square$}
 \end{proof}
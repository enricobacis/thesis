\draft{

With the considerable advancements in IT solutions, users and companies are finding increasingly appealing to rely on external services for storing resources and making them available to others. In such contexts, a promising approach to enforce access control to externally stored resources is via encryption: resources are encrypted for storage and only authorized users have the keys that enable their decryption. There are several advantages that justify the use of encryption for enforcing access control. First, robust encryption has become computationally inexpensive, enabling its introduction in domains that are traditionally extremely sensitive to performance (like cloud-based applications and management of large resources). Second, encryption provides protection against the service provider itself, which - while trustworthy for providing access - cannot typically be considered authorized to know the content of the resources it stores ({\em honest-but-curious\/} scenario) and hence also to enforce access control. Third, encryption solves the need of having a trusted party for policy enforcement: resources enforce self-protection, since only authorized users, holding the keys, will be able to decrypt them.

One of the complex aspects in using encryption to enforce access control policy concerns access revocation. If granting an authorization is easy (it is sufficient to give the newly authorized user access to the key), revoking an authorization is a completely different problem. There are essentially two approaches to enforce revocation: {\em i)\/} re-encrypt the resource with a new key or {\em ii)\/} revoke access to the key itself. Re-encryption of the resource entails, for the data owner, downloading the resource, decrypting it and re-encrypting it with a new key, re-uploading the resource, and re-distributing the key to the users who still hold authorizations. If decryption, re-encryption, and even key management (for this specific context) can be considered a trivial issue, the remaining challenge is represented by the need to download and re-upload the resource, with a considerable overhead for the data owner. This overhead will continue to grow as usage of cloud resources grows, in particular in the context of emerging big data applications. The alternative approach of enforcing revocation on the resource by preventing access to the key with which the resource is encrypted cannot be considered a solution. As a matter of fact, it protects the key, not the resource itself, and it is inevitably fragile against a user who - while having been revoked from an access - has maintained a local copy of the key.

A clear recent trend in information technology is the rent by many
users and enterprises of the storage/computation services from other
parties. With cloud technology, what was in the past managed
autonomously now sees the involvement of servers, often in an unknown
location, immediately reachable wherever an Internet connection is
present.  Today the use of these Internet services typically assumes
the presence of a Cloud Service Provider (CSP) managing the service.
There are a number of factors that explain the current status. In
general, the procurement and management of IT resources exhibit
significant scale economies, and large-scale CSPs can provide services
at costs that are less than those incurred by smaller players. Still,
many users have an excess of computational, storage, and network
capacity in the systems they own and they would be interested in
offering these resources to other users in exchange of a rent
payment. In the classical behavior of markets, the existence of an
infrastructure that supports the meeting of supply and demand for IT
services would lead to a significant opportunity for the creation of
economic value from the use of otherwise under-utilized resources.


This change of landscape is witnessed by the increasing attention of
the research and development community toward the realization of {\em
	Decentralized Cloud Storage\/} (DCS) services, characterized by the
availability of multiple nodes that can be used to store resources in
a decentralized manner. In such services, individual resources are
fragmented in {\em shards\/} allocated (with replication to provide
availability guarantees) to different nodes.  Access to a resource
requires retrieving all its shards.  The main characteristics of a DCS
is the cooperative and dynamic structure formed by independent nodes
(providing a multi-authority storage network) that can join the
service and offer storage space, typically in exchange of some reward.
This evolution has been facilitated by blockchain-based technologies
providing an effective low-friction electronic payment system
supporting the remuneration for the use of the service.  On platforms
such as {\em Storj}~\cite{wilkinson2014storj}, {\em SAFE Network
	Vault}~\cite{irvine2010maidsafe,paul2014security}, {\em
	IPFS}~\cite{benet2014ipfs}, and {\em Sia}~\cite{vorick2014sia},
users can rent out their unused storage and bandwidth to offer a
service to other users of the network, who pay for this service with a
network crypto-currency~\cite{sc-distributed-content-delivery}.


However, if security concerns and perception of (or actual) {\em loss
	of control} have been an issue and slowing factor for centralized
clouds, they are even more so for a decentralized cloud storage, where
the dynamic and independent nature of the network may hint to a
further decrease of control of the owners on where and how their
resources are managed.  Indeed, in centralized cloud systems, the CSP
is generally assumed to be honest-but-curious and is then trusted to
perform all the operations requested by authorized users (e.g., delete
a file when requested by the owner)~\cite{hbcm02}. The CSP is
discouraged to behave maliciously, since this would clearly impact its
reputation. On the contrary, the nodes of a decentralized system may
behave maliciously when their misbehavior can provide economic
benefits without impacting reputation (e.g., sell the content of
deleted files).  Client-side encryption typically assumed in DCSs
provides a first crucial layer of protection, but it leaves resources
exposed to threats, especially in the long term.  For instance,
resources are still vulnerable in case the encryption key is exposed,
or in case of malicious nodes not deleting their shards upon the
owner's request to try reconstructing the resource in its entirety.


Protection of the encryption key is therefore not sufficient in DCS
scenarios, as it remains exposed to the threats above. A general
security principle is to rely on more than one layer of defense.  In
this chapter, we propose an additional and orthogonal layer of
protection, which is able to mitigate these risks.

On the positive side, however, we note that the decentralized nature
of DCS systems also increases the reliability of the service, as the
involvement of a collection of independent parties reduces the risk
that a single malfunction can limit the accessibility to the stored
resources. In addition to this, the independent structure
characterizing DCS systems - if coupled with effective resource
protection and careful allocation to nodes in the network - makes them
promising for actually strengthening security guarantees for owners
relying on the decentralized network for storing their data.

Trusted notaries~\cite{10.1007/BFb0032349} or services~\cite{rabin2006time} are not suitable for all the scenarios~\cite{Abelson:1997:RKR:275079.275104}, as the data owner has to completely entrust a third party.
Therefore, cryptographers have been designing encryption techniques, known as Time-Lock Encryption (TLE)~\cite{may1993timed}, that replace a trusted notary with Time-Lock {\em Puzzles}~\cite{mahmoody-tl,Bitansky:2016:TPR:2840728.2840745}.
In the TLE setting, the data are encrypted so that in order to recover the plaintext, it is required to execute the decryption algorithm $N$ times.
During the encryption process, the sender can tune $N$ to her needs.
Hence, as long as the time to run the decryption algorithm does not change significantly, it is possible to time-lock data for arbitrarily long periods of time.
Thus, TLE can be used to create Time-Locks, yet it is worth noting that the disclosure time is not fixed, but relative to the start of the decryption process.

\bigskip

\bigskip

\section{Document Structure}


This thesis is organized in five chapters.

\smallskip

\paragraph*{Chapter~\ref{chap:intro}} illustrates the scenario, depicts the problem statement and provides a general overview of the thesis overview and contributions.

\paragraph*{Chapter~\ref{chap:waont}}
describes an encryption mode, named \name, that enables efficient access revocation on resources stored at external cloud providers. By leveraging a technique known as {\em All-Or-Nothing Transform}, \name allows the data owner to perform access revocation by re-encrypting a small portion of the file, while providing guarantees about the \ldots

\smallskip

The chapter is organized as follows.

\begin{itemize}
	\item Section~\ref{ms:sec:intro} presents the overall scenario, discusses the trust assumptions, together with existing solutions and their limitations.
	
	\item Section~\ref{ms:sec:mixslice} details the \name encryption mode., an approach to produce an encrypted representation with the desired protection guarantees.
	\item Section~\ref{ms:sec:revoke} presents the enforcement of access revocation. 
	
	\item Section~\ref{ms:sec:security} discusses the effectiveness of our solution in providing revocation.
	
	\item Section~\ref{ms:sec:expe} illustrates our implementation and the extensive experimental evaluation confirming its advantages and applicability. 
	
	\item Section~\ref{ms:sec:relwork} discusses related work. 
	
	\item Section~\ref{ms:sec:conclu} presents our conclusions.
\end{itemize}

\medskip

\paragraph*{Chapter~\ref{chap:dcs}}
presents an approach enabling resource owners to effectively protect and securely delete their resources while relying on decentralized cloud services for their storage.
Our solution combines All-Or-Nothing-Transform for strong resource protection, and carefully designed strategies for slicing resources and for their decentralized allocation in the storage network.  We address both availability and security guarantees, jointly considering them in our model and enabling resource owners to control their setting.

\smallskip

The chapter is organized as follows.

\begin{itemize}
\item Section~\ref{dcs:sec:background} introduces the basic concepts.
\item Section~\ref{dcs:sec:allocation} defines the properties of a decentralized allocation function with respect to replication and protection.
\item Section~\ref{dcs:sec:functions} discusses slicing and allocation strategies.
\item Section~\ref{dcs:sec:analysis} illustrates availability and security guarantees and discusses the setting of parameters guiding slicing and allocation.
\item Section~\ref{dcs:sect:implementation} illustrates the implementation of our approach on a real DCS service and presents experimental results.
\item Section~\ref{dcs:sec:relwork} discusses related work.
\item Section~\ref{dcs:sec:conclusion} concludes the chapter. 
\item The proofs of theorems are provided in Appendix~\ref{dcs:sec:proofs}.
\end{itemize}

\medskip

\paragraph*{Chapter~\ref{chap:ityt}} presents a novel way of implementing time-locked secrets. Time-Locks enable the release of secret data at a specific future point in time. Our vision is that this will be a fundamental piece to be able to fully exploit the power of decentralized cloud storage. The use of time-locked secrets enables the creation of delegated challenge-response protocols, which can be used to provide properties such as access revocation and resiliency in decentralized cloud storage networks.
%
Current research efforts mostly bind the recovery of the secret with the solution of cryptographic puzzles. These solutions, however, are  impractical: not only they require the interested parties to undergo a significant computational effort in order to solve the puzzle, but also they provide no precise timing guarantees.
%
To address these problems, we propose {\em I {\em Told} You Tomorrow (ITYT)}, a novel way of implementing time-locked secrets based on smart contracts to remove the need of any trusted party. We leverage blockchains as a distributed source of time, and we combine threshold cryptography with economic incentives (or penalties) to effectively replace cryptographic puzzles.

\smallskip

The chapter is organized as follows.

\begin{itemize}
	
\item Section~\ref{sect:introduction}
	
\item Section~\ref{sect:background} introduces the basic concepts we build our proposal on. 

\item Section~\ref{sect:model} describes the protocol from a high-level perspective.

\item Section~\ref{sect:constraints} illustrates how to impose constraints on the parameters to protect the resulting protocol from rational adversaries.

\item Section~\ref{sect:realization} describes the details of our reference implementation based on the Ethereum blockchain and the FRESCO sMPC framework.

\item Section~\ref{sect:impl_attacks} discusses the possible attacks the implementation can be subject to, and the respective countermeasures.

\item Section~\ref{sect:evaluation} illustrates the experimental evaluation confirming its applicability.

\item Section~\ref{sect:relwork} discusses related work.

\item Section~\ref{sect:conclusions} draws the conclusions.
\end{itemize}

\medskip

\paragraph*{Chapter~\ref{chap:conclusions}} draws the conclusions of the thesis and presents possible future work.
%In this chapter we also illustrate how the results obtained in Chapter~\ref{chap:ityt} can be used to improve the schema described in Chapter~\ref{chap:dcs}. 

}

\new{

\bigskip

%\section{Contributions}
%The contributions of this document can be summarized as follows.
%\begin{itemize}
%	\item 
%\end{itemize}
%
%\bigskip

\section{Publications}

This sections presents the list of publications authored during the PhD course and that set the basis for this thesis.
			
			
\subsubsection*{Articles in Journals}
\begin{itemize}
	\item Enrico Bacis, Sabrina De Capitani di Vimercati, Sara Foresti, Stefano Paraboschi,	Marco Rosa,	Pierangela Samarati. ``\textbf{Securing Resources in Decentralized Cloud Storage}''. {\em IEEE Transactions on Information Forensics and Security, vol. 15, n. 1}. 2019.
\end{itemize}

\subsubsection*{Papers in Proceedings of International Conferences}
\begin{itemize}	
	
	\item Enrico Bacis, Sabrina de Capitani di Vimercati, Sara Foresti, Stefano Paraboschi, Marco Rosa, Pierangela Samarati. ``\textbf{Access Control Management for Secure Cloud Storage}''. {\em 12th EAI International Conference on Security and Privacy in Communication Networks (SECURECOMM)}. EAI, 2016.
	
	\item Enrico Bacis, Sabrina De Capitani di Vimercati, Sara Foresti, Daniele Guttadoro, Stefano Paraboschi, Marco Rosa, Pierangela Samarati, Alessandro Saullo. ``\textbf{Managing Data Sharing in OpenStack Swift with Over-Encryption}''. {\em 3rd Workshop on Information Sharing and Collaborative Security (WISCS)}. ACM, 2016.
	
	\item Enrico Bacis, Sabrina de Capitani di Vimercati, Sara Foresti, Stefano Paraboschi, Marco Rosa, Pierangela Samarati. ``\textbf{Mix\&Slice: Efficient Access Revocation in the Cloud}''. {\em 23rd ACM Conference on Computer and Communications Security (CCS)}. ACM, 2016.
	
	%\item Enrico Bacis, Sabrina De Capitani di Vimercati, Sara Foresti, Stefano Paraboschi, Marco Rosa, Pierangela Samarati, ``\textbf{Distributed Shuffle Index in the Cloud: Implementation and Evaluation}''. {\em Proceedings of the 4th IEEE International Conference on Cyber Security and Cloud Computing (CSCloud)}. IEEE, 2017.
	
	%\item Enrico Bacis, Alan Barnett, Andrew Byrne, Sabrina De Capitani di Vimercati, Sara Foresti, Stefano Paraboschi, Marco Rosa, Pierangela Samarati. ``\textbf{Distributed Shuffle Index: Analysis and Implementation in an Industrial Testbed}''. {\em Proceedings of the 5th IEEE Conference on Communications and Network Security (CNS)}. IEEE, 2017.
	
	%\item Enrico Bacis, Marco Rosa, Ali Sajjad. ``\textbf{EncSwift and Key Management: An Integrated Approach in an Industrial Setting}''. {\em 3rd Workshop on Security and Privacy in the Cloud (SPC)}. IEEE, 2017.
	
	%\item  Enrico Bacis, Sabrina de Capitani di Vimercati, Dario Facchinetti, Sara Foresti, Giovanni Livraga, Stefano Paraboschi, Marco Rosa, Pierangela Samarati. ``\textbf{Multi-Provider Secure Processing of Sensors Data}''. {\em Proceedings of the 17th International Conference on Pervasive Computing and Communications (Percom)}. IEEE, 2019.
	
	\item Enrico Bacis, Sabrina De Capitani di Vimercati, Sara Foresti, Stefano Paraboschi, Marco Rosa, Pierangela Samarati, ``\textbf{Dynamic Allocation for Resource Protection in Decentralized Cloud Storage}''. {\em Proceedings of the 2019 IEEE Global Communications Conference (GLOBECOM)}. IEEE, 2019.
	
	\item Enrico Bacis, Dario Facchinetti, Marco Rosa, Marco Guarnieri, Stefano Paraboschi, ``\textbf{I {\em Told} You Tomorrow: Practical Time-Locked Secrets using Smart Contracts}''. {\em Under review at the time of submission}.
\end{itemize}

\subsubsection*{Chapters in Books}
\begin{itemize}
	\item  Enrico Bacis,	Sabrina De Capitani di Vimercati, Sara Foresti,	Stefano Paraboschi,	Marco Rosa,	Pierangela Samarati. ``\textbf{Protecting Resources and Regulating Access in Cloud-Based Object Storage}''. {\em From Database to Cyber Security}. 2018.
\end{itemize}

} % end \new
\lipsum[1]

\medskip

\section{Document Structure}

This thesis is organized in five chapters. This first chapter illustrates the problem statement and provides a general overview of the thesis contributions.

\paragraph*{Chapter~\ref{chap:waont}}
describes an encryption mode, named \name, that enables efficient access revocation on resources stored at external cloud providers. By leveraging a technique known as {\em All-Or-Nothing Transform}, \name allows the data owner to perform access revocation by re-encrypting a small portion of the file, while providing guarantees about the \ldots

\smallskip

The chapter is organized as follows.

\begin{itemize}
	\item Section~\ref{ms:sec:intro} presents the overall scenario, discusses the trust assumptions, together with existing solutions and their limitations.
	
	\item Section~\ref{ms:sec:mixslice} details the \name encryption mode., an approach to produce an encrypted representation with the desired protection guarantees.
	\item Section~\ref{ms:sec:revoke} presents the enforcement of access revocation. 
	
	\item Section~\ref{ms:sec:security} discusses the effectiveness of our solution in providing revocation.
	
	\item Section~\ref{ms:sec:expe} illustrates our implementation and the extensive experimental evaluation confirming its advantages and applicability. 
	
	\item Section~\ref{ms:sec:relwork} discusses related work. 
	
	\item Section~\ref{ms:sec:conclu} presents our conclusions.
\end{itemize}

\medskip

\paragraph*{Chapter~\ref{chap:dcs}}
presents an approach enabling resource owners to effectively protect and securely delete their resources while relying on decentralized cloud services for their storage.
Our solution combines All-Or-Nothing-Transform for strong resource protection, and carefully designed strategies for slicing resources and for their decentralized allocation in the storage network.  We address both availability and security guarantees, jointly considering them in our model and enabling resource owners to control their setting.

\smallskip

The chapter is organized as follows.

\begin{itemize}
\item Section~\ref{dcs:sec:background} introduces the basic concepts.
\item Section~\ref{dcs:sec:allocation} defines the properties of a decentralized allocation function with respect to replication and protection.
\item Section~\ref{dcs:sec:functions} discusses slicing and allocation strategies.
\item Section~\ref{dcs:sec:analysis} illustrates availability and security guarantees and discusses the setting of parameters guiding slicing and allocation.
\item Section~\ref{dcs:sect:implementation} illustrates the implementation of our approach on a real DCS service and presents experimental results.
\item Section~\ref{dcs:sec:relwork} discusses related work.
\item Section~\ref{dcs:sec:conclusion} concludes the chapter. 
\item The proofs of theorems are provided in Appendix~\ref{dcs:sec:proofs}.
\end{itemize}

\medskip

\paragraph*{Chapter~\ref{chap:ityt}} presents a novel way of implementing time-locked secrets. Time-Locks enable the release of secret data at a specific future point in time. Our vision is that this will be a fundamental piece to be able to fully exploit the power of decentralized cloud storage. The use of time-locked secrets enables the creation of delegated challenge-response protocols, which can be used to provide properties such as access revocation and resiliency in decentralized cloud storage networks.
%
Current research efforts mostly bind the recovery of the secret with the solution of cryptographic puzzles. These solutions, however, are  impractical: not only they require the interested parties to undergo a significant computational effort in order to solve the puzzle, but also they provide no precise timing guarantees.
%
To address these problems, we propose {\em I {\em Told} You Tomorrow (ITYT)}, a novel way of implementing time-locked secrets based on smart contracts to remove the need of any trusted party. We leverage blockchains as a distributed source of time, and we combine threshold cryptography with economic incentives (or penalties) to effectively replace cryptographic puzzles.

\smallskip

The chapter is organized as follows.

\begin{itemize}
	
\item Section~\ref{sect:introduction}
	
\item Section~\ref{sect:background} introduces the basic concepts we build our proposal on. 

\item Section~\ref{sect:model} describes the ITYT protocol from a high-level perspective.

\item Section~\ref{sect:constraints} illustrates how to impose constraints on the parameters to protect the resulting protocol from rational adversaries.

\item Section~\ref{sect:analysis} presents an analysis of our protocol based on game theory.

\item Section~\ref{sect:realization} describes the details of our reference implementation based on the Ethereum blockchain and the FRESCO sMPC framework.

\item Section~\ref{sect:evaluation} illustrates the experimental evaluation confirming its applicability.

\item Section~\ref{sect:relwork} discusses related work.

\item Section~\ref{sect:conclusions} draws the conclusions.
\end{itemize}

\medskip

\paragraph*{Chapter~\ref{chap:conclusions}} draws the conclusions of the thesis and presents possible future work in the area. In this chapter we also illustrate how the results obtained in Chapter~\ref{chap:ityt} can be used to improve the schema described in Chapter~\ref{chap:dcs}. 

\clearpage

\section{Publications}

This sections presents the list of publications that pose the basis for this thesis.

\begin{itemize}
\item Enrico Bacis, Sabrina De Capitani di Vimercati, Sara Foresti, Stefano Paraboschi, Marco Rosa, Pierangela Samarati. ``Access Control Management for Secure Cloud Storage''. {\em 12th EAI International Conference on Security and Privacy in Communication Networks (SECURECOMM)}. EAI, 2016.
\item Enrico Bacis, Sabrina De Capitani di Vimercati, Sara Foresti, Daniele Guttadoro, Stefano Paraboschi, Marco Rosa, Pierangela Samarati, Alessandro Saullo. ``Managing Data Sharing in OpenStack Swift with Over-Encryption''. {\em 3rd Workshop on Information Sharing and Collaborative Security (WISCS)}. ACM, 2016.
\item Enrico Bacis, Sabrina De Capitani di Vimercati, Sara Foresti, Stefano Paraboschi, Marco Rosa, Pierangela Samarati. ``Mix\&Slice: Efficient Access Revocation in the Cloud''. {\em 23rd ACM Conference on Computer and Communications Security (CCS)}. ACM, 2016.
\item Enrico Bacis, Sabrina De Capitani di Vimercati, Sara Foresti, Stefano Paraboschi, Marco Rosa, Pierangela Samarati, ``Distributed Shuffle Index in the Cloud: Implementation and Evaluation''. {\em Proceedings of the 4th IEEE International Conference on Cyber Security and Cloud Computing (CSCloud)}. IEEE, 2017.
\item Enrico Bacis, Marco Rosa, Ali Sajjad. ``EncSwift and Key Management: An Integrated Approach in an Industrial Setting''. {\em 3rd Workshop on Security and Privacy in the Cloud (SPC)}. IEEE, 2017.
\item Enrico Bacis, Alan Barnett, Andrew Byrne, Sabrina De Capitani di Vimercati, Sara Foresti, Stefano Paraboschi, Marco Rosa, Pierangela Samarati. ``Distributed Shuffle Index: Analysis and Implementation in an Industrial Testbed''. {\em Proceedings of the 5th IEEE Conference on Communications and Network Security (CNS)}. IEEE, 2017.
\item Enrico Bacis, Sabrina De Capitani di Vimercati, Sara Foresti, Stefano Paraboschi, Marco Rosa, and Pierangela Samarati. ``Protecting Resources and Regulating Access in Cloud-Based Object Storage''. {\em From Database to Cyber Security}. Springer, 2018.
\item Enrico Bacis, Sabrina De Capitani di Vimercati, Dario Facchinetti, Sara Foresti, Giovanni Livraga, Stefano Paraboschi, Marco Rosa, Pierangela Samarati, ``Multi-Provider Secure Processing of Sensors Data'', {\em Proceedings of the 17th IEEE International Conference on Pervasive Computing and Communications}. IEEE, 2019.
\item Enrico Bacis, Sabrina De Capitani di Vimercati, Sara Foresti, Stefano Paraboschi, Marco Rosa, Pierangela Samarati. ``Securing Resources in Decentralized Cloud Storage''. {\em IEEE Transactions on Information Forensics and Security}. IEEE, 2019.
\item Enrico Bacis, Dario Facchinetti, Marco Rosa, Marco Guarnieri, Stefano Paraboschi, ``I {\em Told} You Tomorrow: Practical Time-Locked Secrets using Smart Contracts''. {\em (under revision at the time of writing)}.
\end{itemize}

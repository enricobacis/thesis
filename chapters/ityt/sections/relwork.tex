\section{Related Work}\label{sect:relwork}

The use of cryptography to solve the problem of unveiling private data at a specific time in the future was first envisioned in 1993 by Timothy May~\cite{may1993timed}. Since then many researchers have proposed solutions to this problem.
%
Based on the assumptions and technologies used, we can classify the proposals into four main categories.

\smallskip
{\em Trust and Honesty ---} Chan et al.~\cite{1437112} and Cheon et al.~\cite{10.1007/11889663_17} proposed single point of trust schemes, in which users can encrypt secrets using public keys, while the decryption keys are released at predefined times by a trusted time server.
%
Rabin et al. described Time-lapse cryptography~\cite{rabin2006time,PARKES2008294} that overcomes the single point of trust (and failure) assumption by splitting the single authority into a group of users that have to cooperate to release the keys. Li et al.~\cite{dht} proposed a solution that relies on Distributed Hash Tables to route the secret among peers. Both these proposals entail the peers to be honest as they do not consider the possible economic benefits that the participants would obtain by colluding.

\smallskip
{\em Time-Lock Puzzles ---} The second category requires the recipient to solve an inherently sequential mathematical puzzle to prove the elapse of time. Starting from Rivest et al. seminal work~\cite{Rivest:1996:TPT:888615}, many other puzzles have been proposed~\cite{Bitansky:2016:TPR:2840728.2840745,cohen2018,mahmoody-tl}. All these techniques are not practical as the required time might differ on different machines, and the computation costs for the recipient are significant.

\smallskip
{\em Smart Contracts ---} Similarly to our proposal, the third category leverages smart contracts~\cite{szabo1997formalizing} to replace the trusted party. Kimono~\cite{kimono,kimono-network} and Keep Network~\cite{keep} rely on threshold cryptography to split the secret among participants that can earn a remuneration by keeping their shares private until disclosure time. However, they do not introduce security deposits, thus failing at preventing misbehaviors.
%
Li et al. proposal~\cite{self-emerging1} overcomes some of these limitations by modeling the protocol as an extensive-form game with imperfect information~\cite{leyton2008essentials}. Yet, as each peer is a single point of failure, and as the owner has perfect information about the shares, they require every participant to pay a security deposit that exceeds the value of the secret, limiting the applicability of the protocol.
%
Compared to our protocol, all the proposals in this category do not consider the fact that malicious coalitions could reconstruct the secret ahead of time inside an sMPC protocol without exposing the shares, thus effectively avoiding penalties and safeguarding remunerations.

\smallskip
{\em Witness Encryption ---} The last category of solutions leverages witness encryption~\cite{Garg:2013:WEA:2488608.2488667}, in which the sender can encrypt a message so that it can only be opened by a recipient who knows a witness to an NP relation. Liu et al.~\cite{jager2018} showed how to construct a computational reference clock from large public computations, such as those made by the Bitcoin network, and couple it with witness encryption to realize a time-lock encryption mechanism. Yet, due to the lack of practical witness encryption schemes, their proposal is still far from being concrete.

\smallskip
{\em Other Contributions ---}
Several recent proposals try to address the problem of dealing with secret data on public blockchains. Enigma~\cite{enigma} and Hawk~\cite{hawk} leverage sMPC to allow multiple actors to execute an algorithm on private inputs and store the proof of correct execution on the blockchain.
However, these proposals require the data holder to actively participate in the computation, thus they can not be used to solve the problem of data disclosure at a future point in time.
%
{\em Proof of Elapsed Time} ({\em PoET}) is a network consensus algorithm often used in permissioned blockchains, like Hyperledger Sawtooth~\cite{poet,hyperledger-sawtooth} that avoid wasting computational resources by using a fair lottery system run inside a Trusted Execution Environment (TEE), such as Intel SGX. Each participant runs an algorithm in the TEE that waits for a random amount of time, thus proving the elapse of time without the need of PoW. Even if this approach resembles \shortname, as it prevents cheating on the chosen time, it is not able to store secret data.
%
Another recent contribution that showed how economic constraints enforced by time-lock primitives can be successfully integrated with blockchains is the {\em Bitcoin Lightning Network}~\cite{poon2016bitcoin}. Lightning Network can be used to instantly exchange bitcoins among peers by using off-chain transactions while effectively preventing any possible misbehavior.
\section{Economic model}\label{sect:constraints}


This section illustrates how to constraint the protocol parameters to achieve the desired TL scheme behavior.

\paragraph*{Preamble}
As previously mentioned, \shortname assumes all the participants to be rational agents. 
This means they are driven by economic interests and they will always try to maximize their profit. 
So, there is no interest in \secret itself, but only in its economic value.
%
Also, as \shortname is built upon secret sharing, it is pivotal to analyze the scenario by referencing to the behavior of groups of adversaries, and not of single users.
Thus, we reference by \coalition a generic malicious coalition of users that team up to break the TL ahead of disclosure time.

\paragraph*{Method}
To limit the attack surface, we model \shortname as a negative sum game to all the possible adversarial coalitions. 
In other words, we impose that anyone who tries to break the TL schema has to face costs greater than the maximum achievable revenues. 
To ensure this condition, we develop a set of constraints among the economic parameters.
% so that any rational agent that takes part to the protocol will comply with its rules.
However, to define each of them we need to identify the maximum revenue an attacker can achieve.
For this purpose, we focus on the best possible attack scenario, or rather the one in which an adversarial coalition ideally completes its entire strategy without interference by other parties.

Before going into details, we fix the first trivial constraint.
It simply captures, in a single expresion, the relation among the amounts discussed in the previous section:

\begin{equation}\label{eqbase}
\Wshare < \BH < \RH < \Wsecret
\end{equation}

In the following we address how coalitions could try to attack the protocol based on the role of its members and the time at which the attack could be performed.
For the sake of clarity, Subsections~\ref{sect:mal_sha} and ~\ref{sect:mal_own} do not take into account the \texttt{\algowhistleblowshare} function, which will be discussed in as special case in Subsection~\ref{sect:impact_wh}.
Finally, Section~\ref{sect:economic_po} addresses how to constraint \PO.


%In the description we gave so far only the basic protocol operating scheme was presented, but no mention was made about the ratio between the economic constraints that have to be imposed on the parameters.
% The protocol is in fact vulnerable because it is exposed to coalitions of attackers that can be both {\em internal} and {\em external} to the TL mechanism.
%However, since the economic (dis-)incentives are the only way to achieve the compliance of rules, it is required to identify the targets which these rules are addressed to, so that the desired outcome is obtained. In particular, the target of the described protocol is strictly time dependent, as the participants have to protect the secrecy of \secret before the disclosure time, and as after \td they have to make sure it to be disclosed.

\begin{algorithm}[t]
	\caption{SC function to disclose the share after \td}\label{algo:disclose}
	\begin{algorithmic}[1]
		\MyAlgBlock{input}
		\Desc{$sc$}{\hspace*{1.5em}smart contract identifier}
		\Desc{$i$}{\hspace*{1.5em}index of the submitted share}
		\Desc{$\share_i$}{\hspace*{1.5em}the $i$-th share}
		\vspace{0.6em}
		\Procedure{Disclose}{$sc, i, \share_{i}$}
		\If{$sc.\state = \statelocked$ \textbf{and} $sc.\states[i] = \statepaid$}
		\If{$\primtime \geq sc.\td$ \textbf{and} $\primtime < sc.\te$}
		\If {$\primhash{\share_i} = sc.\Cshare{}[i]$}
		\State $sc.\shares \left[ i \right] \gets \share_{i}$
		\State $sc.\numdisclosed\ += 1$
		\State $sc.\states[i]\gets \statedisclosed$
		\State $p_1 \gets sc.\numdisclosed \ge sc.\K$
		\State $p_2 \gets \primtime > \te$
		\State $p \gets p_1 \wedge p_2$
		\State $\primwithdraw{sc}{caller}{\RH}{p}$
		\EndIf
		\EndIf
		\EndIf
		\EndProcedure
	\end{algorithmic}
\end{algorithm}

\subsection{Protection against malicious shareholders}\label{sect:mal_sha}

Being the shareholder rational agents, they will consider if is worth to try breaking the TL before \td or not.
To perform a successful attack, a shareholder has to team up with other $\K - 1$ ones in order to reconstruct \secret, sell it to a buyer, and finally whistleblow it.
In that event, \coalition would earn at most by selling and by whistleblowing,\footnote{The coalition \coalition could setup an additional external contract between \coalition and the buyer to be sure to gain both \V and \Wsecret.} leading the protocol to failure. 
The alternative, i.e. do not break the TL and submit the $k$ shares after \td, would lead to a gain equal to the sum of the rewards. 
To avoid the first scenario we can impose:
%
%\begin{equation}\label{eqbalance1}
%\tag{$1^*$}
$\K \cdot \RH  > \V + \Wsecret$.
%\end{equation}

Yet, in order to to earn $\K \cdot \RH$, the coalition \coalition should wait until \td, whereas $\V + \Wsecret$ could be collected earlier.
For this reason, in \shortname we use a stricter formulation of the constraint, obtained by comparing the revenue earned by whistleblowing and the total amount of bids already paid \coalition to get the shares:

\begin{equation}\label{eqcoscom2}
\K \cdot \BH > \V + \Wsecret
\end{equation}

Constraint (\ref{eqcoscom2}) addresses secrecy (time $t < \td$), however, when the disclosure time passes, the objective of the TL turns into facilitating the disclosure.
In this setting, threshold cryptography may obstacle the release of \secret, offering additional opportunities to malicious shareholders and coalitions of them.
In fact, $\N - \K + 1$ shareholders could wait for $\K - 1$ others to submit their share and then lead the TL instance to a stall by refusing to submit their ones and wait for a buyer willing to pay \V to gain access to the secret.
To avoid this issue, we introduce the constraint:  

\begin{equation}\label{eqcoscom4}
(\N - \K + 1) \cdot \RH  > \V
\end{equation}

Here the contribution of the term $\Wsecret$ disappears as the role of whistleblower is no longer admissible after \td. The inequality holds because the shareholders are authorized to collect their rewards \RH only in case the TL results in a successful outcome\footnote{An alternative approach to reduce stalls without the need of a termination time and deferred rewards would be to grant an extra reward \extrareward to the first \K shareholders that submit the share after \td.
%Timely payment of rewards would encourage this type of attack, as in some situations \coalition could use it to reduce the entry fee while still not publicly disclose the secret (i.e., not write it to the smart contract).
%Such a bonus --- which in total amounts to $\K \cdot \extrareward$ --- would be paid from \PO.
%The latter would contribute positively to inequalities~\ref{eqbalance1} and ~\ref{eqcoscom2}.
%anyhow foreseeing race conditions is not strictly compulsory.
}
(i.e., at least \K shares are submitted before the termination time \te).


\subsection{Protection against malicious owners}\label{sect:mal_own}

All the considerations made about the rationality of the shareholders must also be applied to the owner.
As \owner knows the secret in advance, she could setup fake TL instances for the sole purpose of invoking the \texttt{\algowhistleblowsecret} function and obtain \Wsecret, if this would end up having a positive economic return.
%
%Anyone in possession of the secret can act as a whistleblower, then the owner could be interested in setting up a fake protocol to get $\Wbonus_{\secret}$.
To enforce a negative outcome for this scenario, we need to add the following constraint:

\begin{equation}\label{eqbalance3}
\PO > \Wsecret
\end{equation}

%Since the penalties are enforced by smart contracts, someone might be tempted to prevent the owner from executing the whistleblow protocol, thus replacing~\ref{eqbalance3} by an identity check.  Yet, there is no way to forbid the owner from teaming up with a shareholder, and forming an malicious mixed coalition.  Such a case should be addressed by the addition of:

%\begin{equation}\label{eqbalance7}
%\PO + \BH > \Wbonus_{\secret}
%\end{equation}

%which is a less stringent constraint than~\ref{eqbalance3}.


\subsection{Impact of share whistleblowing function}\label{sect:impact_wh}

The ability to whistleblow shares, which is needed to prevent them to be sold, opens up to more complex strategies that can be performed by coalitions.
A coalition \coalition
% composed by \K shareholders 
could in fact submit some of the controlled shares to the contract before whistleblowing the secret, in order to maximize its revenue.

As the coalition is composed by rational agents, it is possible to determine the optimal number \jopt of shares that \coalition could submit before incurring into penalties or advantaging other participants.
%Indeed, submitting $k$ shares before \td leads to the TL failure, and therefore the impossibility of earning $\Wbonus_{\secret}$, which we remind is greater than $\Wbonus_{\share}$. 
Whistleblowing a share is in fact a public event, observable by anyone, so it could itself trigger other participants' strategies.
%Another aspect we point out is that a share whistleblow is an event observable by all the participants, as the state smart contract is publicly readable. 

The values \N and \K, the number of total shareholders and the threshold, identify two cases to compute  \jopt. \newline \vspace*{-0.5em} \newline
{\em General case.}
In general, a \K-shareholders coalition \coalition is not the only one able to break the TL. 
Each time a share is whistleblown the threshold \K is weakened, and all other $\N-\K$ shareholders and their possible coalitions are triggered as a side effect.
%Let's assume that a first coalition $\coalition ^\prime $ whistleblows a share;
%$\coalition ^\prime $ would earn $\Wshare$, while losing the bid \BH paid to obtain the share.
%
%, accordingly a second coalition $\coalition ^{\prime \prime} $ might form.
%, $\coalition ^{\prime \prime} $ would save \BH in case of an hypothetical \secret reconstruction.
%As $\BH > \Wbonus_{\share}$, then $\coalition ^{\prime \prime} $ would become the most dangerous attacker. \\	%A second coalition $\coalition ^{\prime \prime}$ that only consists of $\K-1$ shareholders would now be able to reconstruct the secret. 
Overall, when $i$ shares have been whistleblown, a quiescent coalition $\coalition^i$ formed by $\K-i$ shareholders, would gain the ability to reconstruct the secret \secret having paid only $(\K-i) \cdot \BH$ to get its shares. 
For this reason, the initial coalition \coalition formed by \K shareholders needs to determine the optimal number of shares \jopt to be whistleblown so that there is no other coalition that can profit from the weakening of \K. In particular, $\coalition$ can determine \jopt by solving the following:	
$$\jopt = \operatorname*{max}_i\ \{i | (\K-i) \cdot \BH > \V + \Wsecret,\ i \in 1,\ldots,\K-1 \}$$
%, while benefit from a cost saving of $i \cdot \BH$ while performing the attack. So, if $i_{inv}$ so that
%$$
%\begin{cases}
%\left( \K - i_{inv} \right) \cdot \BH < \V + \Wbonus_{\secret} \\
%\left( \K - i_{inv} + 1 \right) \cdot \BH > \V + \Wbonus_{\secret} 
%\end{cases}
%$$
If such a \jopt exists, by keeping the assumption of rational agents, the coalition \coalition can submit \jopt shares while still being sure that no other smaller coalition will whistleblow the secret.	\newline \vspace*{-0.5em} \newline
{\em Special case $\K > \lfloor \N/2 \rfloor. $}
In this case, the \K-shareholders coalition \coalition is the only one able to break the TL. 
Specifically, $2\K-\N-1$ shares can be submitted while still preventing the other $\N-\K$ shareholders to reconstruct \secret. 
Please note that such number of submissions could be smaller compared to the one identified by the general case, which is only cost dependent. 
To compute the optimal number of submissions, it is then required to select:
$$ \jopt = \operatorname*{max}\ \{ \left( 2\K - \N - 1 \right) ; \text{ } \jopt _\text{general\textunderscore case} \}$$

Whistleblowing shares permits to gain the extra-revenue $ \Wer = \jopt \cdot \Wshare$.
%
To address it, we formulate a stricter version of Equation~(\ref{eqcoscom2}) by constraining:

\begin{equation}\label{eqcoscom2w}
\K \cdot \BH > \V + \Wsecret + \Wer
\end{equation}


\subsection{Evaluating Costs}\label{sect:economic_po}

Now that we've discussed how to constraint the parameters to prevent misbehavior, we discuss the additional requirements to be satisfied by \PO.
In fact, it is from this amount the owner pays the shareholder profits, or the whistleblower bonuses. 
Overall, the sum of \PO and the shareholder bids $\N \cdot \BH$ must be able to cover the costs for every possible evolution of the TL instance.
The easiest evolution is represented by all the shareholders correctly executing the protocol. To accommodate for this case, we need to impose that the amount stored in the smart contract is enough to pay the shareholder rewards:

\begin{equation}\label{eqpo1}
\PO + \N \cdot \BH \geq \N \cdot \RH
\end{equation}

On the contrary, the TL could fail with $\K-1$ shares whistleblown before the secret itself is whistleblown.
To ensure the smart contract has enough value to be able to pay all the whistleblower bonuses, we impose the following constraint:

\begin{equation}\label{eqpo2}
\PO + \N \cdot \BH \geq (\K-1) \cdot \Wshare + \Wsecret
\end{equation}

As $\BH > \Wshare$, we do not need to discuss mixed evolutions of the TL. In fact, right side of Inequality~(\ref{eqpo2}) reports the worst case amount.
In any other circumstance (i.e., less than $\K-1$ shares whistleblown), 
%disclosure rewards are replaced by whistleblower bonuses, but being $\RH > \Wshare$, 
the Inequality~(\ref{eqpo2}) still holds.

\begin{figure}[t]
	\centering
	\begin{tabular}[b]{| c | c | c | c | c | c |}	
		\hline
		\V & \Wshare & \BH & \RH & \Wsecret & \PO \\
		\hline \hline
		1 & 0.025 &  0.275 &  0.3 &  0.325 &  0.35 \\
   		\hline \hline
		10 & 0.25  &  2.75  &  3   &  3.25  &  3.5  \\
   		\hline \hline
		100 & 0.5   & 25.5   & 26   & 26.5   & 27    \\
		\hline
	\end{tabular}
	\caption{Sample configurations with $\K=5$ and $\N=8$ and $\V \in {1,10,100}$ that respect all the constraints (optimized for minimizing \PO)}
	\label{table:variables}	
\end{figure} 

The constraints (\ref{eqbase}-\ref{eqpo2}) must all hold for any well-formed instance of \shortname. 
They identify the acceptance area for the variables. 
The owner may desire to minimize \PO, while the shareholders would like to maximize the profit $\RH - \BH$.
A more thorough analysis of the ratio among this variables and the different strategies to set them is out of scope in this chapter.
In Table~\ref{table:variables} we show three possible configurations for $\K=5$, $\N=8$, and $\V \in {1, 10, 100}$, which minimize \PO, obtained by constraint programming.
It is worth to remark that in all of the cases we have $\PO < \V$, which is a desirable property.



We present an approach to enforce access revocation on resources stored at external cloud providers. The approach relies on a resource transformation that provides strong mutual inter-dependency in its encrypted representation. To revoke access on a resource, it is then sufficient to update a small portion of it, with the guarantee that the resource as a whole (and any portion of it) will become unintelligible to those from whom access is revoked. The extensive experimental evaluation on a variety of configurations confirmed the effectiveness and efficiency of our solution, which showed excellent performance and compatibility with several implementation strategies


Decentralized Cloud Storage services represent a promising opportunity
for a different cloud market, meeting the supply and demand for IT
resources of an extensive community of users.  The dynamic and
independent nature of the resulting infrastructure introduces security
concerns that can represent a slowing factor towards the realization
of such an opportunity, otherwise clearly appealing and promising for
the expected economic benefits.  In this chapter, we present an approach
enabling resource owners to effectively protect and securely delete
their resources while relying on decentralized cloud services for
their storage. Our solution combines All-Or-Nothing-Transform for
strong resource protection, and carefully designed strategies for
slicing resources and for their decentralized allocation in the
storage network.  We address both availability and security
guarantees, jointly considering them in our model and enabling
resource owners to control their setting.


Time-Locks enable the release of secret data at a specific future point in time.
Current research efforts mostly bind the recovery of the secret with the solution of cryptographic puzzles.
These solutions, however, are  impractical: 
not only they require the interested parties to undergo a significant computational effort in order to solve the puzzle, but also they provide no precise timing guarantees.

To address these problems, we propose ITYT, a novel way of implementing time-locked secrets based on smart contracts to remove the need of any trusted party.
We leverage blockchains as a distributed source of time, and we combine threshold cryptography with economic incentives (or penalties) to effectively replace cryptographic puzzles.

%This way, clients can be involved in multiple TL instances without wasting computational resources.

We implement a prototype of our approach on top of the Ethereum blockchain. Our prototype leverages secure Multi-Party Computation to avoid any single point of trust.
We also analyze resiliency to attacks with the help of economic games theory.
The experiments demonstrate the low costs and resource consumption associated with our approach.  



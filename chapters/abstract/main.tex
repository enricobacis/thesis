The low costs and high reliability guarantees associated with cloud storage led
many IT organizations to offload their data to cloud service providers. Yet,
this raises new challenges to manage access control and data confidentiality in
these environments.

The first part of this doctoral thesis analyzes centralized cloud storage
providers (CSP). In this setting, the CSP is considered honest-but-curious: it
always complies with users' requests, but it might access unprotected data. A
possible defense is to encrypt the data; however, standard encryption modes
would introduce relevant overheads when performing access revocation. To
address this problem, we present an approach that relies on a resource
transformation that provides strong mutual inter-dependency in its encrypted
representation. To revoke access on a resource, it is then sufficient to update
a small portion of it.

The second part studies how these guarantees can be extended to the
decentralized cloud storage environment. In this case, the data is sharded and
offloaded in a peer-to-peer network, in which nodes might be dishonest and try
to disobey users' deletion and access revocation requests to maximize their
revenue. We propose a solution that addresses both availability and security
guarantees and enables resource owners to tune these settings to their needs.

When dealing with decentralized networks, an important aspect is how to detect
misbehaving nodes, to stop paying for their service and migrate the data to new
peers. This process has to work even when the data owner is offline and without
imposing trust or honesty assumption on any of the involved parties. To address
this problem, in the third part of this thesis, we detail a novel way of
deploying self-releasing time-locked secrets. This technique can be used to
implement delegated challenge-response protocols that, in turn, can guarantee
data confidentiality and retrievability properties in decentralized systems.
Our solution leverages smart contracts and economic incentives to regulate a
game among the mutually distrusting users that compose a blockchain, thus
removing the need of any trusted party.

The technologies detailed in this thesis push the state of the art as regards
resource protection and access regulation in centralized and decentralized
cloud storage systems. The implementations have been released under
open-source licenses and can be readily integrated with real systems.

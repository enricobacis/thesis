\chapter[Mix\&Slice Implementation]{Mix\&Slice Implementation}

This appendix describes the open source implementation of the \emph{Mix\&Slice} encryption mode described in chapter~\ref{chap:waont}.

The implementation is done in C and consists of single-threaded and a
multi-threaded encryption/decryption functions that make use of AES as
base symmetric encryption primitives. The use of OpenSSL EVP APIs
leverages hardware-accelerated AES-NI primitives when available.

\section{APIs}\label{usage}

The file \texttt{includes/aes\_mix.h} contains the following three
definitions:

\begin{itemize}
\tightlist
\item
  \texttt{BLOCK\_SIZE}: number of bytes in a cipher block (16 bytes for
  AES).
\item
  \texttt{MINI\_SIZE}: number of bytes in a miniblock.
\item
  \texttt{MINI\_PER\_MACRO}: number of mini-blocks in a macro-block.
\end{itemize}

\emph{These entities can be modified at compile time to experiment with
different sizes.}

\subsection{Single-thread APIs}\label{single-thread-apis}

The file \texttt{includes/aes\_mix.h} contains the prototype of the only
two methods that are necessary to use Mix\&Slice:

\begin{minted}{c}
void mixencrypt(const unsigned char* data, unsigned char* out,
                const unsigned long size, const unsigned char* key,
                const unsigned char* iv);

void mixdecrypt(const unsigned char* data, unsigned char* out,
                const unsigned long size, const unsigned char* key,
                const unsigned char* iv);
\end{minted}

The parameters are as follows.

\begin{itemize}
\tightlist
\item
  \texttt{data}: pointer to the source buffer (plaintext in case of
  \texttt{mixencrypt} and ciphertext in case of \texttt{mixdecrypt})
\item
  \texttt{out}: pointer to the destination buffer
\item
  \texttt{size}: number of bytes in source (and destination) buffers
\item
  \texttt{key}: symmetric key (string) used for the AES functions
\item
  \texttt{iv}: initialization vector for the AES functions
\end{itemize}

\subsection{Multi-thread APIs}\label{multi-thread-apis}

The file \texttt{includes/aes\_mix\_multi.h} contains the prototypes of
the only two methods that are necessary to use Mix\&Slice in
multi-threaded mode:

\begin{minted}{c}
void t_mixencrypt(unsigned int thr, const unsigned char* data,
                  unsigned char* out, const unsigned long size,
                  const unsigned char* key, const unsigned char* iv);

void t_mixdecrypt(unsigned int thr, const unsigned char* data,
                  unsigned char* out, const unsigned long size,
                  const unsigned char* key, const unsigned char* iv);
\end{minted}

The only additional parameter is \texttt{thr}, the number of threads to
use.

\subsection{Slicing phase}\label{slicing-phase}

The mixing phase is the real encryption phase. The slicing phase
strongly depends on the file management and should be implemented
according to the ratio of policy updates with respect to decryption
processes and can be easily sped up with as-hoc file management. Because
of this, the performance of the mixing phase is a good proxy of the
performance of the whole Mix\&Slice technique.

The version implemented here keeps the fragments together. This benefits
the policy update process, whereas the decryption process has to pay the
overhead for rearranging the bytes before performing the unmixing phase.

The file \texttt{includes/aes\_mixslice.h} contains the prototypes of
the two methods that perform the whole Mix\&Slice encryption:

\begin{minted}{c}
void mixslice(unsigned int thr, const unsigned char* data,
              unsigned char* fragdata, const unsigned long size,
              const unsigned char* key, const unsigned char* iv);

void unsliceunmix(unsigned int thr, const unsigned char* fragdata,
                  unsigned char* out, const unsigned long size,
                  const unsigned char* key, const unsigned char* iv);
\end{minted}

The \texttt{mixslice} method first uses \texttt{t\_mixencrypt} to
perform the mixing phase. The slicing phase rearranges the output of the
mixing phase in slices. The user is responsible for creating the buffer
that will contain the \texttt{fragdata}. The slices are concatenated and
written to the \texttt{fragdata} buffer. The user of the function, can
read the fragments directly from there as follows:

\begin{itemize}
\tightlist
\item
  each fragment consists of
  \texttt{fragsize\ =\ size\ /\ MINI\_PER\_MACRO} bytes;
\item
  the first fragment spans the \texttt{fragdata} bytes in range
  \texttt{{[}0,\ fragsize)};
\item
  the second fragment spans the \texttt{fragdata} bytes in range
  \texttt{{[}fragsize,\ fragsize*2)};
\item
  and so on until \texttt{{[}size\ -\ fragsize,\ size)}.
\end{itemize}


\subsection{Installation}

Before proceeding please install the \texttt{openssl/crypto} library
source and the \texttt{libtool} binary. In ubuntu you can proceed as
follows:

\begin{minted}{shell-session}
$ sudo apt install libtool-bin libssl-dev
\end{minted}

To compile and install the dynamic library in your system you can:

\begin{minted}{shell-session}
$ make
$ sudo make install
\end{minted}

To remove the library simply do:

\begin{minted}{shell-session}
$ sudo make uninstall
\end{minted}

\subsection{Test}\label{test}

There are three test suites:

\begin{itemize}
\item
  \emph{main}: main test suite that verifies that Mix\&Slice principles
  are enforced.
\item
  \emph{blackbox}: test suite that verifies the Mix\&Slice principles in
  an \emph{abstract} sense (without knowledge about the code).
\item
  \emph{multithread}: test suite that verifies that the Mix\&Slice
  principles are enforced in the multi-threaded implementation.
\end{itemize}

\emph{make} is used for compilation and testing purposes. A basic
\emph{compile-and-test} setup is made by the steps:

\begin{minted}{shell-session}
$ make
$ make test
\end{minted}

See the \texttt{Makefile} for all the compile and test targets.


\section{Python Wrapper}\label{aesmix-python}

The python implementation
wraps both the phases and offers a CLI tool that wraps the
\texttt{libaesmix} library.

The C implementation has been built with performance in mind, whereas
the python wrapper and the CLI tool has been implemented to offer a more
widespread access of the Mix\&Slice capabilities. The mixing and slicing
phases use the C implementation, but the python conversion adds a big
overhead since it has to materialize all the buffers in memory.

Since the tool materializes all the buffers in memory and has to perform
both the mixing and the slicing phases, you should only use the CLI tool
on files that are at maximum as large as a third of your available
memory.

Please check the file \texttt{example.py} to understand how to use the
library.

\subsection{Requirements}\label{requirements}

Before proceeding please install the \texttt{openssl/crypto} library
source. In ubuntu you can proceed as follows:

\begin{minted}{shell-session}
$ sudo apt install libssl-dev
\end{minted}

\subsection{Installation}\label{installation}

The package has been uploaded to PyPI at \url{https://pypi.org/project/aesmix} so, after installing the requirements, you can install the latest released version using pip:

\begin{minted}{shell-session}
$ pip install aesmix
\end{minted}

To install the version from this repository, you can use the commands:

\begin{minted}{shell-session}
$ make build
$ sudo make install
\end{minted}

To install the package in a virtual environment, use:

\begin{minted}{shell-session}
python setup.py install
\end{minted}

The python wrapper will also compile the \texttt{libaesmix} library.

\subsection{Command Line Interface}\label{command-line-interface}

This package also installs the \texttt{mixslice} tool that can be used
as follows.

To encrypt a file:

\begin{minted}{shell-session}
$ mixslice encrypt sample.txt
INFO: [*] Encrypting file sample.txt ...
INFO: Output fragdir:   sample.txt.enc
INFO: Public key file:  sample.txt.public
INFO: Private key file: sample.txt.private
\end{minted}

To perform a policy update:

\begin{minted}{shell-session}
$ mixslice update sample.txt.enc
INFO: [*] Performing policy update on sample.txt.enc ...
INFO: Encrypting fragment #68
INFO: Done
\end{minted}

To decrypt a file:

\begin{minted}{shell-session}
$ mixslice decrypt sample.txt.enc
INFO: [*] Decrypting fragdir sample.txt.enc using key sample.txt.public ...
INFO: Decrypting fragment #68
INFO: Decrypted file: sample.txt.enc.dec

$ sha1sum sample.txt sample.txt.enc.dec
d3e92d3c3bf278e533f75818ee94d472347fa32a  sample.txt
d3e92d3c3bf278e533f75818ee94d472347fa32a  sample.txt.enc.dec
\end{minted}

\begin{center}\rule{0.5\linewidth}{\linethickness}\end{center}

\section{Key regression mechanism}\label{key-regression-mechanism}

The key regression mechanism implementation is based on
\href{https://eprint.iacr.org/2005/303.pdf}{``Key Regression: Enabling
Efficient Key Distribution for Secure Distributed Storage''}.

\subsubsection{Example}

The key regression mechanism can be used as follows.

\begin{minted}{python}
from aesmix.keyreg import KeyRegRSA


iters = 5
stp = KeyRegRSA()

print("== WINDING ==")
for i in range(iters):
    stp, stm = stp.wind()
    print("k%i: %r" % (i, stm.keyder()))

print("\n== UNWINDING ==")
for i in range(iters - 1, -1, -1):
    print("k%i: %r" % (i, stm.keyder()))
    stm = stm.unwind()
\end{minted}
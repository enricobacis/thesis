\new {

% GENERAL
In this thesis, we presented novel techniques for providing data confidentiality, access control, and availability guarantees to resources stored in centralized, decentralized, and distributed cloud storage systems.
%

% MIX&SLICE
As regards centralized cloud storage providers, we presented an approach for efficiently enforcing access revocation on encrypted resources stored at external providers. Our solution enables data owners to effectively revoke access by re-encrypting a small portion of the resource.
%
As opposed to techniques based on other {\em All-or-Nothing Transforms} in which an actor that has access to the file could try to maintain a local copy of the keys to counter access revocation, {\em Mix\&Slice} enforces that the amount of data that the malicious actor would have to store can not be determined a priori, as the data owner could re-encrypt a varying portion of the resource.
%
Our implementation and experimental evaluation confirm the efficiency and effectiveness of our proposal.

% DCS
We then analyzed how to extend the guarantees that we obtained with {\em Mix\&Slice} to provide effective secure protection to decentralized cloud storage providers.
%
Our approach enables resource owners to protect their resources and to control their decentralized allocation to different nodes in the network. We investigated different strategies for splitting and distributing resources and analyzed them in terms of availability and security guarantees. Our proposal allows the data owner to control the granularity of slicing and diversification of allocation to ensure aimed availability and security guarantees even in presence of malicious coalition of nodes that are willing to disobey the owner's requests to maximize their revenue.

% ITYT
Finally, we presented \name (\shortname), a practical schema to deploy Time-Locked secrets on the blockchain. Our proposal leverages an economic model in which every rational actor (or coalition of them) is economically incentivized to comply with the protocol.
%
This is a fundamental piece to create delegated challenge-response protocols in which the challenges and responses are offloaded in an encrypted form by the data owner and then decrypted at different time.
%
The use of delegate challenge-response protocols permits to remove the coordinator nodes and bring the security and availability guarantees to the fully distributed settings without the need of the data owner or a trusted third party to be online.
}

\new{

\section{Future Work}

We conclude the thesis with a discussion of the future work that can be done in the three considered areas: \textit{centralized}, \textit{decentralized}, and \textit{distributed} storage systems.

\begin{description}
	
	\item[Centralized storage systems] -- Chapter~\ref{chap:waont} illustrated how to use {\em Mix\&Slice} to enable the use. The two implementations discussed in the evaluation section were based on \textit{AES} and \textit{OAEP}. We showed how the use of \textit{OAEP} enables to obtain micro-blocks with size up to 256 bits. We used the \textit{SHA2} class of functions to compute the hashes; however, these functions are not optimized in common CPUs.
	
	As future work, it is possible to compare the results with other functions that also comply with the requirements in Section~\ref{ms:sect:oaep}. A preliminary evaluation with the linear transformations proposed by Karame et al. for \textit{Bastion}~\cite{bastion} and by Naor et al.~\cite{Naor1999} showed promising results.
	
	\item[Decentralized storage systems] -- Chapter~\ref{chap:dcs} described how to bring security and availability guarantees to the decentralized cloud storage scenario. We also briefly mentioned how these properties could be integrated with fountain codes to leverage the dynamicity of DCS networks. In \cite{globecom} we showed how the integration seems to be a good fit to address the dynamicity problem. This idea has yet to be explored in details; Future work in this area consists of implementing and evaluating different codes and protocols to fully exploit the dynamicity of the system.
	
	\item[Distributed storage systems] -- Chapter~\ref{chap:ityt} showed how to deploy time-locked secrets on the blockchain, and briefly discussed how this primitive can be used for delegated challenge-response protocols. Future work in this area can formalize {\em delegated challenge-response} protocols and show how they can be used in several scenarios such as inheritance management, voting processes, autonomous organizations, and Proof-of-Retrievability for cloud storage.
	
	Moreover, as already discussed in the chapter, the use of secure Multi-Party Computation is not the only way to realize a secure share generation and distribution protocol. We are already considering the use of homomorphic encryption, which would further reduce the time and resources required to setup the ITYT protocol.
	
\end{description}

}
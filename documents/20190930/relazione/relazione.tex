\documentclass{article}

\usepackage[italian]{babel}
%\usepackage[T1]{fontenc} 
\usepackage[utf8]{inputenc}
\usepackage{url}

\usepackage{balance}


\begin{document}

\title{Relazione finale di dottorato}
\author{Enrico Bacis}
\date{30/09/2019}

\maketitle

\vspace{40px}

Nel corso del primo anno di dottorato (XXXII ciclo) in Ingegneria e Scienze Applicate presso l’Universita` degli Studi di Bergamo, ho affrontato principalmente il tema della security nell’ambito dei servizi cloud centralizzati.
\\

Durante il secondo anno, abbiamo applicato le tecniche e le conoscenze sviluppate nel corso del primo anno ai sistemi di cloud storage decentralizzati. In questo modo abbiamo potuto studiare ...
\\

Nel corso del terzo anno invece, abbiamo affrontato il problema di come automatizzare il rilascio di informazioni segrete in un istante di tempo futuro. ...

\subsubsection*{Pubblicazioni}

Ho collaborato a quattro pubblicazioni nell’ambito della security in cloud per conferenze internazionali [3, 5, 4, 6].

Ho collaborato a due pubblicazioni nell'ambito della security in cloud per conferenze internazionali: ``{\em Distributed shuffle index: Analysis and implementation in an industrial testbed}'' \cite{8228695} and ``{\em EncSwift and key management: An integrated approach in an industrial setting}'' \cite{8228711}. Un'altra pubblicazione è attualmente in revisione \cite{dcs}.
\\

\subsubsection*{Collaborazione in Progetti Europei}

Grazie alla guida del mio advisor Prof. Stefano Paraboschi, ho avuto la possibilità di collaborare al progetto europeo EscudoCloud (nell’ambito del programma Horizon 2020). Questo mi ha dato l’opportunità di conoscere e lavorare con gli altri partner del progetto, tra i quali vi sono sia partner accademici (Università di Milano e TUD Damstadt) che e industriali (Dell EMC, IBM, SAP, British Telecom, Wellness Telecom).
Il progetto europeo ha coinvolto il nostro gruppo di ricerca nella realizzazione di diversi strumenti open source (disponibili all’indirizzo https://github.com/ escudocloud), nonché nella stesura di alcuni deliverable di progetto [1, 2].

Come nell'anno precedente, grazie alla guida del mio advisor Prof. Stefano Paraboschi, ho avuto la possibilità di collaborare al progetto europeo {\em ESCUDO-CLOUD} (nell'ambito del programma Horizon 2020). Questo mi ha dato l'opportunità di collaborare con gli altri partner del progetto, tra i quali vi sono sia partner accademici (Università di Milano e TUD Damstadt) sia industriali (Dell EMC, IBM, SAP, British Telecom, Wellness Telecom).
Durante questo anno, il progetto europeo ha coinvolto il nostro gruppo di ricerca nella realizzazione di diversi strumenti open source (disponibili all'indirizzo \url{https://github.com/escudocloud}), nonché tre deliverable di progetto \cite{D26, D35, D44}.
\\

\subsubsection*{Periodi di ricerca all'estero}

Nei mesi di Luglio, Agosto e Settembre 2017, ho svolto un periodo di ricerca all'estero a Monaco di Baviera (DE) presso l'ente di ricerca Google Germany GmbH, su tematiche relative alla sicurezza del linguaggio WebAssembly, attualmente supportato da tutti i browser più diffusi.
\\

Nei mesi di Luglio, Agosto, e Settembre 2018, ho svolto un periodo di ricerca a Londra (UK) presso l'ente di ricerca Google UK, su tematiche relative alla sicurezza delle applicazioni Android.
\\

Nei mesi di Luglio, Agosto, e Settembre 2019, ho svolto un periodo di ricerca a Zurigo (CH) presso l'ente di ricerca Google Switzerland GmbH, su tematiche relative alla privacy dei dati degli utenti del web. 


\subsubsection*{Attività di dottorato}

Per quanto riguarda le attività di dottorato e i crediti formativi da raggiungere, allego a questo documento la tabella che riporta la mia situazione finale.

\vspace{20px}

\bibliography{biblio}
\bibliographystyle{plain}

\end{document}